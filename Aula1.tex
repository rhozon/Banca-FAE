% Copyright 2019 Clara Eleonore Pavillet

% Author: Clara Eleonore Pavillet
% Description: This is an unofficial Oxford University Beamer Template I made from scratch. Feel free to use it, modify it, share it.
% Version: 1.0

\documentclass{beamer}
\input{Theme/Packages.tex}
\usepackage{latexsym}
\usetheme{oxonian}






%================CAPA=================================
\title{Variáveis aleatórias contínuas}
\titlegraphic{\includegraphics[width=2cm]{faecapa.png}}
\author{Rodrigo Hermont Ozon}
\institute{Banca de admissão de professor

\textit{FAE Business School}}
\date{04 de novembro de 2019} %\today
%===============fim da capa ==========================



%===============Sumário==============================
\begin{document}

{\setbeamertemplate{footline}{} 
\frame{\titlepage}}

\section*{Conteúdo}\begin{frame}{Conteúdo}\tableofcontents\end{frame}
%========Fim do Sumário==============================

%==========Plano de Aula================

\section{Plano de Aula}
    \begin{frame}[plain]
        \vfill
      \centering
      \begin{beamercolorbox}[sep=8pt,center,shadow=true,rounded=true]{title}
        \usebeamerfont{title}\insertsectionhead\par%
        \color{oxfordblue}\noindent\rule{10cm}{1pt} \\
        \LARGE{\faFileTextO}
      \end{beamercolorbox}
      \vfill
  \end{frame}

%======================================================
\begin{frame}{Plano de Aula}
\vspace{-.5cm}
\footnotesize
\textbf{Pré-Requisitos de conhecimento:} Conhecimento prévio de cálculo de esperança, variância e desvio-padrão para variáveis discretas. Para as contínuas será necessário um conhecimento prévio de cálculo de integrais definidas. 

\textbf{Por quê distinguir variáveis aleatórias e contínuas e momentos da distribuição (valor esperado e variância, desvio-padrão) ?}

\textbf{Objetivo da aula de hoje:}

    \begin{itemize}
        \item Definir variáveis aleatórias contínuas e discretas;
        \item Como medimos esperança (valor médio), variância e desvio-padrão para variáveis discretas e como isso pode ser extendido para as contínuas;
        \item Por quê utilizamos cálculo integral para as contínuas ?
        \item Como calculamos valor esperado, variância e desvio-padrão para as contínuas ?
        \item Fazer alguns exercícios de \textit{cases} reais demonstrando a aplicabilidade dos conceitos ensinados;
        \item Introdução para cálculos de momentos para FDPs de distribuições contínuas conhecidas (próximas aulas)
    \end{itemize}
    

\end{frame}
%=====================================================
\begin{frame}{Plano de Aula}
    
O material para exercícios dos exemplos usados na aula de hoje e exercícios propostos estão disponíveis no meu repositório do GitHub:

\vspace{.5cm}
\fbox{Acesse \href{https://github.com/rhozon/Banca-FAE}{\faGithub}} e baixe os arquivos de dados no formato .xls e também essa apresentação no formato \LaTeX\, (para os nerds de plantão)



\vspace{.5cm}
Estatística é muitíssimo interessante e nada complicada, pois hoje temos os computadores a nosso favor!!!
        
    
\end{frame}











\section{Definido Variáveis Aleatórias}
    \begin{frame}[plain]
        \vfill
      \centering
      \begin{beamercolorbox}[sep=8pt,center,shadow=true,rounded=true]{title}
        \usebeamerfont{title}\insertsectionhead\par%
        \color{oxfordblue}\noindent\rule{10cm}{1pt} \\
        \LARGE{\faFileTextO}
      \end{beamercolorbox}
      \vfill
  \end{frame}

%======================================================

%========Definição de Variáveis Aleatórias Contínuas x Discretas ===========================================
\begin{frame}{Definição}

Uma \textbf{variável aleatória} é aquela cujos valores estão associadas a probabilidades (ou seja razão de chance).

\begin{itemize}
    \item \textbf{Variável Aleatória Discreta} $\Rightarrow$ Ela pode assumir um número finito de diferentes valores dentro de um intervalo finito.
    \textit{''Qualquer função definida sobre o espaço amostral $\Omega$ que atribui um valor real a cada elemento do espaço amostral."}
    \item \textbf{Variável Aleatória Contínua} $\Rightarrow$ É aquela que pode assumir um número infinito de diferentes valores dentro de um intervalo finito. 
    Dizemos que $x$ é uma v.a. contínua que poderá assumir qualquer valor real entre seus valores de mínimo e máximo.
    \textit{"Seja $ X $ uma variável aleatória. Suponha que o contradomínio ($ \mathbb{R}_x $) de $ X $ seja um intervalo ou uma coleção de intervalos. Então diremos que $ X $ é uma variável aleatória contínua.''}%Uma função X definida sobre o espaço amostral Ω e assumindo
%valores num intervalo de números reais, é denominada variável aleatória contínua.

\end{itemize}


\end{frame}
%====================================================




%========Definição de Variáveis Aleatórias Contínuas x Discretas ===========================================
\begin{frame}{Variável aleatória discreta x contínua}

Exemplos de variáveis aleatórias discretas:
\begin{itemize}
    \item Lançamento de uma moeda ou um dado
    \item Contagem do número de pessoas de uma família
    \item Quantidade de alunos aprovados em econometria
\end{itemize}

\vspace{.5cm}

Exemplos de variáveis aleatórias contínuas:
\begin{itemize}
    \item As estaturas dos participantes aqui
    \item Cotação de preços de ações
    \item Distância percorrida
\end{itemize}

\end{frame}
%=====================================================

%=====================================================
\section{Formalização matemática}
    \begin{frame}[plain]
        \vfill
      \centering
      \begin{beamercolorbox}[sep=8pt,center,shadow=true,rounded=true]{title}
        \usebeamerfont{title}\insertsectionhead\par%
        \color{oxfordblue}\noindent\rule{10cm}{1pt} \\
        \LARGE{\faFileTextO}
      \end{beamercolorbox}
      \vfill
  \end{frame}
%=====================================================



%===================Def. Matemática var. discreta x contínua ============================================
\begin{frame}{Variável aleatória discreta x contínua}

\textbf{Propriedades da Discreta:}

Seja $X$ uma v.a. discreta e sejam $\left\{ x_{1}, x_{2}, x_{3},...x_{n} \right\}$ os valores que esta variável pode tomar. A função de probabilidade de X, é uma função

$$
fdp(x_{i})=p(X=x_{i})
$$

que permite calcular as probabilidades de todos os valores de $X$, tal que:
\begin{itemize}
    \item $fdp(x_{i})\geq 0\quad \forall x \in \mathbb{R}\,\,\mbox{pois}\,\,p(\mbox{Evento})=\displayestyle\frac{n(\mbox{Evento})}{n(\Omega)}$
    \item $\displaystyle\sum_{x}=fdp(x)=1$ (a soma de todas as probabilidades ou frequências tem que fechar em 100\%)
\end{itemize}


\end{frame}
%=====================================================



%=====================================================
\begin{frame}{Probabilidades da Discreta}\label{dado}

\footnotesize

\vspace{-.5cm}
\textcolor{red}{Exemplo 1:}

Em um lançamento de um dado quais as probabilidades (frequências) que podem ocorrer de cair em cada uma das faces possíveis ?

% Table generated by Excel2LaTeX from sheet 'Plan1'
\begin{table}[htbp]
  \centering
    \begin{tabular}{cc}
    \textbf{face} & \textbf{prob. (espaço amostral $\Omega$)} \\
    \hline
    1     &  1/6=16,67\% \\
    2     &  1/6=16,67\% \\
    3     &  1/6=16,67\% \\
    4     &  1/6=16,67\% \\
    5     &  1/6=16,67\% \\
    6     &  1/6=16,67\% \\
    \hline
    \end{tabular}%
  \label{tab:addlabel}%
\end{table}%

Como 1/6=16,67\% a chance de cair o dado com face 1 ou 2 será dada pela soma 16,67\%+16,67\%=33,33\%

\vspace{.25cm}
\textbf{*Pergunta:} Qual a distribuição de probabilidade que melhor descreve o lançamento de um dado ? %Resposta: Uma distribuição uniforme discreta X tem distribuição uniforme discreta
%se cada um dos n valores em sua faixa, (x1, . . . , xn) tiver igual probabilidade.Sua função de probabilidade é dada por f(xi) = 1/n
\end{frame}

%====================================================

%=====================================================
\begin{frame}{Probabilidades da Discreta}
\vspace{-.5cm}
\footnotesize

Histograma da função densidade de probabilidade do lançamento de um dado (não-viciado):

\begin{center}
\includegraphics[width=8cm,height=4cm]{histogramadado}
\end{center}

*Calcule a variância de face do dado $=\left\{1,2,3,4,5,6\right\}$ e compare com o valor esperado (esperança) dele. (Lembre-se que Variância amostral da discreta = $var(X)=\frac{\sum^{n}_{i=1}(x_{i}-\overline{x})^2}{n-1}$. Explique com o resultado obtido, porque isso ocorre.(Verifique que a média também será igual a $E(x)=\sum_{i=1}^{n}x_{i}prob_{xi}$)

\end{frame}
%===================================================
\begin{frame}{Prob como limite de uma frequência relativa}

\vspace{-.5cm}
\footnotesize
(Hoffmann, 2006, p. 8 - 9)~\cite{hoffmann} :
\begin{quote}
    Denominamos frequência relativa do evento A ao quociente entre o número de vezes em que A ocorreu ($n_{A}$) e o número total de eventos observados ($n$).
    
    Podemos, então, definir probabilidade de A como o limite da frequência relativa de A, quando o número de eventos observados tende ao infinito, isto é:
    $$
    p(A)=\lim_{n\to\infty}\frac{n_{A}}{n}
    $$
    
    O conceito de probabilidade como limite de uma frequência relativa é bastante útil na prática. As casas de jogos e as companhias de seguros dependem da estabilidade, a longo prazo, de frequências relativas. Entretanto, a definição não é satisfatória do ponto de vista matemático, porque não é possível uma interpretação rigorosa sem usar o próprio conceito de probabilidade.
    
    \textbf{Afirmações como "a probabilidade de que haja uma guerra nuclear no próximo ano é 0,05", que envolvem uma probabilidade subjetiva, não podem ser interpretadas como limites de frequências relativas.} [grifo meu, sic] %diferença entre abordagem frequentista x bayesiana de probs
\end{quote}


\end{frame}
%===================================================
\begin{frame}{Dados tabelados contínuos}

\footnotesize
\vspace{-.5cm}
\textcolor{red}{Exemplo 2}: Um economista aplicou um questionário via formulários do Google para pesquisar o nível de renda dos clientes de uma empresa pela qual ele presta consultoria. Para avaliar se os dados coletados são coerentes com suas expectativas e se existe um padrão estatístico (distribuição de probabilidades das respostas) ele deverá identificar as características da distribuição que gera esses dados e para isso precisará calcular o valor médio, variância e desvio-padrão...

\begin{table}[htbp]
  \centering
    \begin{tabular}{cc}
    \toprule
    \hline
    \multicolumn{1}{l}{\textbf{Renda (em Salários Mínimos)}} & \textbf{Freq. Absoluta} \\
    \hline
    \midrule
    5 a 10  & 5 \\
    10 a 15 & 8 \\
    15 a 20 & 4 \\
    20 a 25 & 3 \\
    \bottomrule
    \hline
    \end{tabular}%
  \label{tab:addlabel}%
\end{table}%

(Lembrando que $E(X)=\sum^{n}_{i=1}p(x)\cdot\mbox{ponto médio}$ e a $Var(X)=E(X^2)-[E(X)]^2$ sendo o desvio padrão a raiz quadrada da variância). Você também poderia avaliar a assimetria e curtose dessa distribuição


\textbf{*Pergunta}: É possível encontrar a probabilidade (freq. relativa) de um indivíduo ganhar específicamente 5,3 salários mínimos ?

\end{frame}

%=====================================================
\begin{frame}{Probabilidades da Contínua}

(Sartoris, 2013 p. 3)~\cite{sartoris}

\vspace{.25cm}
\textit{Sendo o conjunto $X$ definido por $X = \left\{x \in \mathbb{R}\, | 0 < x < 2\right\}$, qual a probabilidade de, ao sortearmos um número qualquer deste conjunto este número pertença ao intervalo $[0,5; 1,5]$ ? }

\footnotesize
\textit{O conjunto $X$ é um conjunto contínuo, já que contém todos os números reais que sejam
maiores do que 0 e menores do que 2. }

\textit{Tem, por exemplo, o número 1; o número 0,5; o número 0,4; mas também tem o 0,45; o 0,475; o 0,46. Dados dois elementos deste conjunto, sempre é possível
encontrar um número que esteja entre estes dois. Não há “saltos” ou “buracos”, daí a idéia de
continuidade. \textbf{Ao contrário do dado em que os valores possíveis são 1, 2, 3, 4, 5 e 6 (não existe 1,5
ou 2,1), que é um conjunto discreto.}}

\textit{Neste caso, a probabilidade de sortearmos qualquer número entre 0,5 e 1,5 (inclusive), que é
um intervalo de comprimento igual a 1 (= 1,5 – 0,5), de um intervalo possível que tem comprimento
igual a 2 (= 2 – 0) será dada por: $p(0,5\leq x \leq 1,5)=1/2$ (ou seja, 1 chance em 2 possíveis)}

\end{frame}

%===================================================


%=====================================================
\begin{frame}{Probabilidades da Contínua}

\footnotesize
\textit{E qual a probabilidade deste número ser exatamente igual a 1 ? }

\textit{Ou seja, de sortear um único número entre um total de números presente no conjunto X de... \textbf{infinitos!} A probabilidade será dada, então por:}

$$
p(x=1)=\lim_{n\to\infty}\frac{1}{n}=0
$$

\textit{Portanto, embora seja possível de ocorrer, a probabilidade de ser igual a 1 (ou igual a
qualquer número) é igual a zero, se estivermos falando de um \textbf{conjunto contínuo}. A probabilidade
só será diferente de zero se estivermos falando de um intervalo contido neste conjunto.}

\textit{Como consequência disso, não fará diferença se o intervalo para o qual encontramos
inicialmente a probabilidade (entre 0,5 e 1,5) fosse fechado ou aberto (isto é, incluísse ou não os
extremos), pois a probabilidade de ser exatamente 0,5 ou 1,5 é zero. Portanto, como X é um
conjunto contínuo:}

$$
p(0,5 \leq x \leq 1,5) = p(0,5 < x < 1,5) =\frac{1}{2}
$$
\end{frame}

%====================================================

%=====================================================
\section{Integrando x Somando para variáveis aleatórias}
    \begin{frame}[plain]
        \vfill
      \centering
      \begin{beamercolorbox}[sep=8pt,center,shadow=true,rounded=true]{title}
        \usebeamerfont{title}\insertsectionhead\par%
        \color{oxfordblue}\noindent\rule{10cm}{1pt} \\
        \LARGE{\faFileTextO}
      \end{beamercolorbox}
      \vfill
  \end{frame}
%=====================================================






%====================================================
\begin{frame}{Exemplo de aplicação de probs em va contínua}

\textcolor{red}{O uso da integral...}

Um pesquisador deseja medir a profundidade de um rio em locais aleatórios, onde $x$ será a medida nesse local específico. Sabemos que existem diferentes pontos do rio onde a profundidade varia. Assim o pesquisador precisa delimitar uma faixa (área) para um provável ponto A (menos profundo) e ponto B (mais profundo). O espaço delimitado pela área compreendida entre A e B será a variável aleatória $x$ de interesse.
%grafico da área

Ele sabe que a área delimitada por essa curva tem que ser igual a 1 (100\%) e isso lhe concederá sua função densidade de probabilidade, portanto temos:

$$
p(a\leq x \leq b)=\displaystyle\int^{b}_{a}f(x)dx=1
$$



    
\end{frame}

%====================================================
\begin{frame}{Exemplo de aplicação de probs em va contínua}
\footnotesize

Após o pesquisador tirar várias faixas de área de igual amplitude (mesmo tamanho), ele estabelece um ponto médio ($x=x_{i}=(b-a)/n$) para cada uma das áreas e agora ele pensa em calcular uma média desses valores:

$$
\sum_{i=1}^{n} \mbox{ponto médio}f(x)h
$$

Ele sabe que quanto menor for a amplitude ($h=x_{n}-x_{i}$) mais precisa será sua média, portanto ele acrescenta um limite:

$$
\lim_{n\to \infty}\displaystyle\sum_{i=1}^{n}\mbox{ponto médio}f(x)h
$$

E se o limite existir então integramos para obter:
$$
\displaystyle\int_{a}^{b}\mbox{ponto médio}f(x)dx
$$

Onde $dx$ substituiria o $h$ da amplitude.

Assim ele descobre como chegar no conceito de valor médio de uma variável aleatória contínua...

\end{frame}
%===================================================
%========Def. Matemática var. discreta x contínua ============================================

\begin{frame}{Variável aleatória discreta x contínua}


\textbf{Propriedades da Contínua}

\footnotesize
Seja $X$ uma variável aleatória tal que:

\begin{itemize}
    \item $fdp(x)\geq 0 \quad \forall x\in \mathbb{R} $
    \item $fdp(x)$ tem no máximo um número finito de descontinuidades em qualquer subintervalo finito de $\mathbb{R}$.
    \item Seja $A \subset \, \mathbb{R}$, a probabilidade ($p$) de $p(x\,\in\, A)=p(A)=\int_{A}fdp(x)dx$
    \item $\displaystyle\int_{-\infty}^{+\infty} fdp(x)dx=1$
    \item $\displaystyle\int_{a}^{b} fdp(x)dx=p(a\leq\,x\,\leq b) $
    \item $\int_{a}^{\infty}fdp(x)dx=\lim_{k\to\infty}\int_{a}^{k}fdp(x)dx$ ou $\int_{-\infty}^{b}fdp(x)dx=\lim_{k\to\infty}\int_{k}^{b}fdp(x)dx$ desde que o limite seja finito.
\end{itemize}
\end{frame}
%===================================================





%=====================================================
\section{Esperança (valor médio), variância e desvio-padrão}
    \begin{frame}[plain]
        \vfill
      \centering
      \begin{beamercolorbox}[sep=8pt,center,shadow=true,rounded=true]{title}
        \usebeamerfont{title}\insertsectionhead\par%
        \color{oxfordblue}\noindent\rule{10cm}{1pt} \\
        \LARGE{\faFileCodeO}
      \end{beamercolorbox}
      \vfill
  \end{frame}
%===================================================== 
\begin{frame}{Características das Distribuições de Probabilidade}


\textit{Uma distribuição de probabilidade pode ser muitas vezes resumida em termos de algumas características, conhecidas como \textbf{momentos} da distribuição. Dois dos momentos mais utilizados são a \textbf{média}, ou \textbf{valor esperado}, e a \textbf{variância}.}

(GUJARATI, 2000 p. 769)~\cite{gujarati}

\end{frame}
%====================================================





%=====================================================
\begin{frame}{Valor Esperado}

Esperança, valor esperado e valor médio são a mesma coisa.

Como medir a esperança de uma discreta e uma contínua ?

Qual a probabilidade de obtermos um valor específico numa amostra de dados discretos e contínuos ?

\begin{itemize}
    \item Na discreta é dada por $p(x)=f(x_{i})=\displaystyle\frac{x_{i}}{\sum_{i=1}^{n}}$
    \item Na contínua, como vimos é nula, pois $p(X=X_{0})=0$
\end{itemize}


\end{frame}
%=====================================================

\begin{frame}{O que é o Valor Esperado ? $E(X)=\sum_{x}x\cdot p(x)$}

\footnotesize
A esperança ou valor esperado é a média ponderada dos valores que a variável aleatória ou função assume, usando-se, como pesos para ponderação, as probabilidades correspondentes a cada valor.

$$
\fbox{E(X)=\sum_{x}x\cdot p(x)\quad \mbox{se v.a. discreta}}
$$

$$
\fbox{E(X)=\displaystyle\int_{-\infty}^{\infty}x\cdot f(x)dx\quad \mbox{se v.a. contínua}}
$$
Nessa definição supõe-se que somatório e a integral convergem. Em caso contrário dizemos que o valor esperado da variável aleatória X não existe.

\textit{"A única diferença entre este caso e o valor esperado de uma va discreta é que substituímos o símbolo do somatório pelo símbolo da integral."}
(Gujarati, 2000 p. 770)~\cite{gujarati}

\end{frame}
%=====================================================
\begin{frame}{Propriedades dos valores esperados}

\footnotesize

\begin{enumerate}
    \item<1-4> O valor esperado de uma constante é a própria constante. Assim, se $b$ for uma constante, $E(b)=b$
    \item<2-4> Se $a$ e $b$ forem constantes,
    $$
    E(aX+b)=aE(X)+b
    $$
    \item<3-4> Se $X$ e $Y$ forem variáveis aleatórias independentes\footnote{Na Teoria das Probabilidades, duas variáveis aleatórias são independentes quando a ocorrência duma não é influenciada pela ocorrência da outra.}, então
    $$
    E(XY)=E(X)E(Y)
    $$
    \item<4-4> Se $X$ for uma variável aleatória com FDP denotada por $f(x)$ e se $g(X)$ for uma função qualquer de $X$, então
$$
E[g(X)]=\displaystyle\sum_{x}g(X)\cdot f(x)\cdot\quad\mbox{se x for uma va discreta}
$$

$$
E[g(X)]=\displaystyle\int^{\infty}_{-\infty}g(X)\cdot x\cdot f(x)dx\quad\mbox{se for contínua}
$$
\end{enumerate}
    
\end{frame}
%=====================================================

%=====================================================
%\begin{frame}{Exercício valor médio (esperança)}

%Obtenha o valor esperado para a seguinte amostra:

%\begin{table}[htbp]
%  \centering
%    \begin{tabular}{cc}
%    \hline
%    \textbf{x} & \textbf{p(x)} \\
%        \midrule
%        \hline
%    -2    & 27\% \\
%    0     & 12\% \\
%    2     & 26\% \\
%    3     & 35\% \\
%    \hline
%    \end{tabular}%
%  \label{tab:addlabel}%
%\end{table}%

%$$
%E(X)=\displaystyle\sum_{x}x\cdot %p(x)=-2\cdot27\%+0\cdot12\%+2\cdot26\%+3\cdot35\%
%$$

%\end{frame}
%===================================================== 
\begin{frame}{Exercício valor médio (esperança)}\label{Exercvalmedio}
\footnotesize
Considere a seguinte função densidade de probabilidade:

$$
f(x)=\frac{1}{9}x^{2};\quad 0\leq x \leq 3
$$

Obtenha seu valor esperado.


Resolução: Primeiramente avaliamos se $f(x)$ é uma FDP

$$
fdp(x)= \displaystyle\int_{0}^{3}\frac{1}{9}x^{2}dx=1
$$

Resolvendo a integral temos:
$$
\frac{1}{9}\cdot \int _0^3x^2dx\Rightarrow\frac{1}{9}\left[\frac{x^{2+1}}{2+1}\right]^3_0\Rightarrow \frac{1}{9}\left[\frac{x^3}{3}\right]^3_0\Rightarrow\frac{1}{9}\cdot\left[\frac{3^3}{3}\right]-\frac{1}{9}\cdot\left[\frac{0^3}{3}\right]\Rightarrow\frac{1}{9}\cdot 9\Rightarrow 1
$$

ou seja, a $f(x)$ é uma FDP!
 
 
Se quisermos avaliar a FDP entre 0 e 1, fazemos $$
\displaystyle\int^{1}_{0}\frac{1}{9}x^{2}dx=\frac{1}{9}\cdot \left[\frac{x^3}{3}\right]^1_0=1/27=3,7\%
$$

Ou seja, a probabilidade de $x$ se encontrar entre 0 e 1 é de 1/27.
\end{frame}
%====================================================
\begin{frame}{Exercício valor médio (esperança)}

Agora, para obtermos o \fbox{valor esperado} da FDP contínua fazemos:
$$
E(X)=\displaystyle\int^{3}_{0}x\left(\frac{1}{9}x^{2}\right)
$$

$$
\frac{1}{9}\cdot\displayestyle\int_{0}^{3}xx^{2}=\frac{1}{9}\cdot \int _0^3x^3dx
$$

$$
=\frac{1}{9}\left[\frac{x^{4}}{4}\right]^3_0=\frac{1}{9}\cdot \frac{3^4}{4}-\frac{1}{9}\cdot\frac{0^4}{4}=\frac{9}{4}=\fbox{2,25}
$$


\end{frame}
%=====================================================
\begin{frame}{Exercício Esperança da Contínua}

\footnotesize
\textcolor{red}{Exemplo 3}

Um economista deseja construir um modelo de projeção dos preços futuros da opção de uma ação. Ele sabe que os valores (em centavos) das cotas da opção variaram entre 0 e 1 dólar durante um bom tempo e essa é a expectativa do mercado. Ele estima um modelo econométrico para a função da curva de uma faixa de tempo e obtém a seguinte função:
$$
\widehat{Y}=f(x)=1,5(1-x)^2\quad\mbox{com os valores oscilando numa faixa entre 0 e 1}
$$

Ele deseja saber qual o valor médio (provável) dos valores da cotação se encontrarem na faixa entre 0 e 1 dólar...

\end{frame}
%=====================================================
\begin{frame}{Resolução}
\footnotesize
1. Verifica se $f(x)$ é uma FDP igualando a 1:
$$
f(x)=\int _0^11.5\left(1-x^2\right)dx = 1
$$    
$$
=1.5\cdot \int _0^11-x^2dx\Rightarrow 1.5\left(\int _0^11dx-\int _0^1x^2dx\right)
$$
Integral de uma constante $\int adx=ax$   
 $$
 =1.5\left(\left[ x \right]_{0}^{1} - \left[\frac{x^3}{3}\right]^{1}_{0} \right)\Rightarrow 1.5\left(1-\frac{1}{3}\right)\Rightarrow 1 \quad \mbox{logo a f(x) é uma FDP!}
 $$
    
    
\end{frame}
%=====================================================
\begin{frame}{Resolução (continuação)}

\begin{picture}    
\centering
\caption{Área sobre a curva da FDP: $f(x)=\displaystyle\int _0^11.5\left(1-x^2\right)dx = 1$}
\includegraphics[width=7cm,height=7cm]{areacurvaopcoes}

\end{picture}

\end{frame}
%====================================================
\begin{frame}{Resolução (continuação)}
\footnotesize
2. Como f(x) é uma FDP podemos obter o valor médio simplesmente inserindo $x$ antes da função dentro da integral definida:
$$
f(x)=\int _0^1 x1.5\left(1-x^2\right)dx = 1
$$

$$    
=1.5\cdot \int _0^1x\left(1-x\right)^2dx
$$

Aplicando a fórmula do quadrado perfeito  
$$ \left(a-b\right)^2=a^2-2ab+b^2\Rightarrow 1^2-2\cdot \:1\cdot \:x+x^2\Rightarrow x\left(x^2-2x+1\right)$$
$$
=x^3-2x^2+x$
$$

Aplicando a regra da soma:
$$
=1.5\left(\int _0^1xdx-\int _0^12x^2dx+\int _0^1x^3dx\right)\Rightarrow 1.5\left(\left[\frac{x^2}{2}\right]^{1}_{0}-\left[2\frac{x^3}{3}\right]^{1}_{0}+\left[\frac{x^4}{4}\right]^{1}_{0}\right)
$$


\end{frame}
%======================================================
%=====================================================
\begin{frame}{Resolução (continuação)}
\footnotesize

$$
1.5\left(\left[\frac{x^2}{2}\right]^{1}_{0}-\left[2\frac{x^3}{3}\right]^{1}_{0}+\left[\frac{x^4}{4}\right]^{1}_{0}\right)=
$$

$$
=1.5\left(\frac{1}{2}-\frac{2}{3}+\frac{1}{4}\right)\Rightarrow 0,125
$$

Ou seja, o valor médio correspondente a um intervalo de probabilidade compreendido entre 0 e 1 para a oscilação do preço da opção é de 0,125 centavos de dólar.
\end{frame}
%=====================================================
\begin{frame}{Variância}
\footnotesize
Seja $X$ uma variável aleatória e $E(X)=\mu$. A distribuição ou dispersão, dos valores $X$ em torno do valor esperado pode ser medida pela variância, que é definida como:
$$
var(X)=\sigma^2_X=E(X-\mu)^2
$$

A raiz quadrada positiva de $\sigma^2_X$,$\sigma_X$ é definida como desvio-padrão de $X$. A variância ou desvio-padrão dão indicação de quão próximo ou dispersamente os valores $X$ individuais se espalham em torno do seu valor médio. A variância definida anteriormente é calculada como segue:

$$
var(X)=\displaystyle\sum_{x}(X-\mu)^2 f(x)\quad\mbox{se X for uma va discreta}
$$

$$
var(X)=\displaystyle\int^{\infty}_{-\infty} (X-\mu)^2 f(x) dx=\displaystyle\int_{-\infty}^{\infty}x^2 f(x)dx\quad\mbox{se X for uma va contínua}
$$
Por conveniência de cálculo, a fórmula da variância dada anteriormente também pode ser expressa como
    $$
    var(X)=\sigma_X^2=E(X-\mu)^2\Rightarrow   E(X^2)-\mu^2\Rightarrow  var(X)=E(X^2)-[E(X)]^2
    $$
  
\end{frame}
%====================================================
\begin{frame}{Propriedades da variância}
    \begin{enumerate}
        \item $E(X-\mu)^2=E(X^2)-\mu^2$
        \item a variância de uma constante é zero
        \item Se $a$ e $b$ forem constantes, então
        $$
        var(aX+b)=a^2 var(X)
        $$
        \item Se $X$ e $Y$ forem variáveis aleatórias independentes, então
        $$
        var(X+Y)=var(X)+var(Y)
        $$
        $$
        var(X-Y)=var(X)+var(Y)
        $$
        \item Se $X$ e $Y$ forem va independentes e $a$ e $b$ forem constantes, então
        $$
        var(aX+bY)=a^2 var(X)+b^2 var(Y)
        $$
    \end{enumerate}


\end{frame}
%====================================================
\begin{frame}{Exercício de variância e desvio-padrão}
\footnotesize

\textcolor{red}{Exemplo 4}
Considere a seguinte FDP:

\tiny
\begin{table}[htbp]
  \centering
    \begin{tabular}{cc}
    \hline
    \textbf{$x$} & \textbf{$f(x)$} \\
    \hline
    -2    & 0,625 \\
    1     & 0,125 \\
    2     & 0,25 \\
    \hline
    \end{tabular}%
  \label{tab:addlabel}%
\end{table}%
\footnotesize

Para sabermos a variância dessa discreta fazemos:
$$
var(X)=E(X^2)-[E(X)]^2
$$
$$
E(X)=-2\cdot0,625+1\cdot0,125+2\cdot0,25=-0,625
$$
Então temos $E(X)=-0625$ e $E(X^2)=3,625$ logo $var(X)=3,625- (-0,625)^2\Rightarrow \fbox{3,23}$

\tiny
\begin{table}[htbp]
  \centering
    \begin{tabular}{ccrcc}
    \hline
    \textbf{x} & \textbf{f(x)} &       & \textbf{x\^2} &  \\
    \hline
    -2    & 0,625 &       & 4    \\
    1     & 0,125 &       & 1     \\
    2     & 0,25  &       & 4      \\
    \hline
    \end{tabular}%
  \label{tab:addlabel}%
\end{table}%

\footnotesize
Como o desvio-padrão é dado pela raiz quadrada da variância obtemos $\sqrt{var(X)}\Rightarrow \sqrt{3,23}\Rightarrow \fbox{\sigma_X=2,06}$

\end{frame}
%======================================================
\begin{frame}{Exercício variância da va contínua}
  
 Obtenha a variância para a função densidade:
$$
f(x)=\frac{1}{9}x^{2};\quad\ 0 \leq x \leq 3
$$ 
    
\textcolor{red}{Solução:} 

Como vimos no Exemplo \textcolor{blue}{\ref{Exercvalmedio}} que essa função é uma FDP, pois o valor da integral se iguala a 1, partimos aplicando a fórmula:

$Var(X)=\displaystyle\int^{b}_{a}x^2 f(x)dx$


Obtemos $var(X)=\displaystyle\int_{0}^{3}x^2 \left(\frac{1}{9}x^2\right) dx$ temos que:

$$
var(X)=\frac{1}{9}\cdot \int _0^3x^2x^2dx\Rightarrow \frac{1}{9}\cdot \int _0^3x^4dx \Rightarrow \frac{1}{9}\left[\frac{x^{4+1}}{4+1}\right]^3_0\ \Rightarrow \frac{1}{9}\left[\frac{x^5}{5}\right]^3_0
$$
$$
var(X)=\frac{1}{9}\cdot \frac{243}{5}\Rightarrow \frac{27}{5}\Rightarrow \fbox{5,4}
$$
\end{frame}
%===================================================
\begin{frame}{Lista de Exercícios}

Clique no link abaixo e baixe o e-book com as páginas e exercícios indicados a seguir:

\vspace{.5cm}
\fbox{\textcolor{blue}{\href{https://github.com/rhozon/Banca-FAE/blob/master/\%5BP.\_A._Morettin\%2C\_W.\_de\_O.\_Bussab\%5D\_Estat\_stica\_B\_si.pdf}{Morettin \& Bussab, Estatística Básica. 6a edição, São Paulo, 2010.}}}


\begin{itemize}
    \item pág 167 (Fazer os exercícios 1, 3 e 4) (encontre as FDP)
    \item pág 172 (Fazer o exercício 9)
    \item pág 173 (Fazer os exercícios 10 e 12)
\end{itemize}

\end{frame}
%===================================================    

%=============Contato=================================
\subsection{Meus contatos}
\begin{frame}{Contacte-me}

\begin{center}
Esta apresentação (no formato \LaTeX{}) e o material recomendado está disponível no meu repositório do GitHub
 \href{https://github.com/rhozon/Banca-FAE}{\faGithub}
\end{center}


Você também poderá acessar meu CV e portfólio em:
\begin{center}
\vspace{.25cm}
\fbox{\textcolor{blue}{\href{https://rhozon.github.io/}{https://rhozon.github.io/}}}
\vspace{.25cm}
\end{center}

\begin{itemize}
    \item Meu repositório no GitHub \href{https://github.com/rhozon}{\faGithub}
    \item Meu perfil no LinkeDin \href{https://www.linkedin.com/in/rodrigohermontozon/}{\faLinkedin}
    \item Meu Currículo na \href{http://lattes.cnpq.br/3532649625879285}{\textcolor{blue}{Plataforma Lattes}}
    \item \href{https://public.tableau.com/profile/rodrigoozon#!/}{Minha página no \textcolor{blue}{Tableau}}
    \item \href{https://api.whatsapp.com/send?phone=5541988382904&text=Ol\%C3\%A1\%20prof.\%20Rodrigo\%2C\%20participei\%20de\%20sua\%20banca\%20na\%20FAE...}{\textcolor{blue}{me mande um zapzap ...}}
    \subitem ou então \textcolor{blue}{\href{mailto:rodrigoozon@yahoo.com.br}{me envie um e-mail}}
\end{itemize}

\vspace{.025cm}
\begin{center}
    Mutíssimo obrigado a todos! \faSmileO
\end{center}
\end{frame}

%=====================================================

%=====================================================
\begin{frame}{Referências}
\footnotesize
\begin{thebibliography}{}

\bibitem{sartoris} SARTORIS, A. \textbf{Estatística e Introdução à Econometria}. Quarta edição. São Paulo, 2013

\bibitem{hoffmann} HOFFMANN, R. \textbf{Estatística para economistas}. Quarta edição, São Paulo, 2006.

\bibitem{gujarati} GUJARATI, D.N., PORTER, D.N. \textbf{Econometria Básica.} Quinta Edição, São Paulo, 2011. Disponível em \href{https://github.com/rhozon/Livros-Econometria/blob/master/Econometria\%20-\%20Damodar\%20N.\%20Gujarati\%20e\%20Dawn\%20C.\%20Porter.pdf}{github.com/rhozon/}

\end{thebibliography}
\end{frame}
%=====================================================







%=====================================================
\end{document}
%=====================================================
